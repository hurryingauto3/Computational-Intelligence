%----------------------------------------------------------------------------------------
%	PACKAGES AND THEMES
%----------------------------------------------------------------------------------------
\documentclass[aspectratio=169,xcolor=dvipsnames]{beamer}
\usetheme{SimplePlus}

\usepackage{algorithm,algorithmic}
\usepackage{hyperref}
\usepackage{graphicx} % Allows including images
\usepackage{booktabs} % Allows the use of \toprule, \midrule and \bottomrule in tables

\renewcommand{\algorithmicrequire}{\textbf{Input:}}
\renewcommand{\algorithmicensure}{\textbf{Output:}}

%----------------------------------------------------------------------------------------
%	TITLE PAGE
%----------------------------------------------------------------------------------------

\title[short title]{A multi-objective adaptive evolutionary algorithm to extract communities in networks} % The short title appears at the bottom of every slide, the full title is only on the title page

\author[Haris-Ali] {Haris Karim Ladhani \& Ali Hamza}

\institute[HU-CI] % Your institution as it will appear on the bottom of every slide, may be shorthand to save space
{
    CS 451 - Computational Intelligence \\
    Habib University % Your institution for the title page
}
\date{\today} % Date, can be changed to a custom date


%----------------------------------------------------------------------------------------
%	PRESENTATION SLIDES
%----------------------------------------------------------------------------------------

\begin{document}

\begin{frame}
    % Print the title page as the first slide
    \titlepage
\end{frame}

\begin{frame}{Overview}
    % Throughout your presentation, if you choose to use \section{} and \subsection{} commands, these will automatically be printed on this slide as an overview of your presentation
    \tableofcontents
\end{frame}

%------------------------------------------------
\section{Introduction}
%------------------------------------------------

\begin{frame}{Introduction}
\begin{itemize}
    \item Many complex systems such as social networks, protein networks can be abstracted into networks i.e. large graphs
    \item Analysis techniques of these networks require studying various patterns within them
    \item Common patterns include recognizing substructures within the graphs such as cliques
    \item This paper analyzes a social network data set and attempts to find communities within it via a genetic algorithm to reduce the complexity of the problem
    \item Communities can be defined by various definitions therefore the paper uses two fitness functions.
    \item \textit{max(internal links) and min(external links)}
\end{itemize}

\end{frame}

\begin{frame}{Problem Description}
    The problem at hand from a CS theoretical point of view is simply a clique, or sub graph, finding problem. 
    %How to find a clique in a graph
    %Clique finding - formal algorithm
    %NP Complete
\end{frame}

\begin{frame}{Motivation}
    %Genetic algorithm part of FYP 
    %Looking for cliques in protient networks
    %Opportunity to explore non-protein network solutions
\end{frame}

%------------------------------------------------

%------------------------------------------------
\section{Genetic Algorithm Formulation}
%------------------------------------------------

\begin{frame}{Encoding Scheme}
    
\end{frame}

%------------------------------------------------
\begin{frame}{Chromosome Representation}
    
\end{frame}

%------------------------------------------------

\begin{frame}{Selection}
   
\end{frame}

%------------------------------------------------

\begin{frame}{Crossover}
   
\end{frame}

%------------------------------------------------

\begin{frame}{Mutation}
   
\end{frame}

%------------------------------------------------

\begin{frame}{Fitness Function}
   
\end{frame}

\begin{frame}{Elitism}
    
\end{frame}
%------------------------------------------------

\begin{frame}{Algorithm}
    \begin{algorithm}[H]
        \begin{algorithmic}[1]
        \STATE \textbf Input: Adaptive parameters: adaptive crossover probability $p_{c}$ of
        population $P$, adaptive mutation probability $p_{m}$ of population $P$
        \STATE $P \leftarrow $ Initialize Population
        \STATE $E \leftarrow $ Initialize Elite Gene Pool
        \STATE While Termination != true;
        \STATE $P_{parent} \leftarrow Select(P)$
        \STATE $p_{c}, p_{m} \leftarrow Adaptive();$
        \STATE $p_{cross} \leftarrow Crossover(P_{parent}, p_{c})$
        \STATE $p_{cross} \leftarrow Mutation(P_{cross}, p_{m})$
        \STATE $P \leftarrow Update(P_{child})$
        \STATE Update Elite Gene Pool
        \STATE End;
        \STATE \textbf{Output} The results of community detection //{Transforming the most adaptable non-inferior solutions from the elite gene pool into community detection results}
        \end{algorithmic}
        \caption{Framework of F-SGCD Algorithm}
        \label{alg:seq}
    \end{algorithm}
\end{frame}

%------------------------------------------------

%------------------------------------------------
\section{Experimental Results}
%------------------------------------------------

\begin{frame}{Evaluation Metrics}
    \begin{itemize}
       \item \textbf{Normalized Mutual Information (NMI)}: It is used to measure the similarity between the detected communities and the known communities. Given two partitions $A$ and $B$ of a network in communities, let $C$ be the confusion matrix whose element $C_{ij}$ is the number of nodes of community $i$ of the partition $A$ that are also in the community $j$ of the partition $B$.
       \item \textbf{Modularity}: It is a criterion for evaluating the quality of community detection.
    \end{itemize}
\end{frame}

%------------------------------------------------

\begin{frame}{Experiments on synthetic LFR networks}
    \begin{itemize}
        \item \textbf{Normalized Mutual Information (NMI)}: It is used to measure the similarity between the detected communities and the known communities. Given two partitions $A$ and $B$ of a network in communities, let $C$ be the confusion matrix whose element $C_{ij}$ is the number of nodes of community $i$ of the partition $A$ that are also in the community $j$ of the partition $B$.
        \item \textbf{Modularity}: It is a criterion for evaluating the quality of community detection.
    \end{itemize}
\end{frame}

%------------------------------------------------

\begin{frame}{Experiments on synthetic real-world networks}
    \begin{itemize}
        \item Normalized Mutual Information (NMI) \\
        It is used to measure the similarity between the detected communities and the known communities. Given two partitions $A$ and $B$ of a network in communities, let $C$ be the confusion matrix whose element $C_{ij}$ is the number of nodes of community $i$ of the partition $A$ that are also in the community $j$ of the partition $B$.
        \item Modularity \\
        It is a criterion for evaluating the quality of community detection.
    \end{itemize}
\end{frame}

%------------------------------------------------

\begin{frame}{Network hierarchy of Pareto solution}
    \begin{itemize}
        \item Hierarchal Modularity \\
        It is used to measure the similarity between the detected communities and the known communities. Given two partitions $A$ and $B$ of a network in communities, let $C$ be the confusion matrix whose element $C_{ij}$ is the number of nodes of community $i$ of the partition $A$ that are also in the community $j$ of the partition $B$.
        \item Bottlenose Dolphins \\
        It is a criterion for evaluating the quality of community detection.
    \end{itemize}
\end{frame}

%------------------------------------------------
\section{Conclusion}
%------------------------------------------------

\begin{frame}{Conclusion}
    \begin{itemize}
        \item Normalized Mutual Information (NMI) \\
        It is used to measure the similarity between the detected communities and the known communities. Given two partitions $A$ and $B$ of a network in communities, let $C$ be the confusion matrix whose element $C_{ij}$ is the number of nodes of community $i$ of the partition $A$ that are also in the community $j$ of the partition $B$.
        \item Modularity \\
        It is a criterion for evaluating the quality of community detection.
    \end{itemize}
\end{frame}
%------------------------------------------------

% \begin{frame}{References}
%     % Beamer does not support BibTeX so references must be inserted manually as below
%     \footnotesize{
%         \begin{thebibliography}{99}
%             \bibitem[Smith, 2012]{p1} John Smith (2012)
%             \newblock Title of the publication
%             \newblock \emph{Journal Name} 12(3), 45 -- 678.
%         \end{thebibliography}
%     }
% \end{frame}

%------------------------------------------------

% \begin{frame}
%     \Huge{\centerline{\textbf{The End}}}
% \end{frame}

%----------------------------------------------------------------------------------------

\end{document}