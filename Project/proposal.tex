\documentclass[11pt,letterpaper]{article}
\oddsidemargin 0in
\evensidemargin 0in
\textwidth 6.5in
\topmargin -0.5in
\textheight 9.0in
\usepackage{hyperref}
\usepackage{mathptmx}
\usepackage{graphicx}
\usepackage[usenames,dvipsnames]{xcolor}
\newcommand{\bibent}{\noindent \hangindent 40pt}
\newenvironment{reference}{\newpage \section*{References} }{\newpage }
\newcommand{\blue}[1]{\textcolor{RoyalBlue}{#1}}


\begin{document}

\title{CS451 - Project Proposal}
\author{Ladhani, Haris Karim\\
\texttt{hl04349@st.habib.edu.pk}
\and
Hamza, Ali\\
\texttt{ah05084@st.habib.edu.pk}
\and
Samson, Synclair Saqib\\
\texttt{ss05901@st.habib.edu.pk}
}
\maketitle

\begin{center}
  \emph{Multi-objective Evolutionary Inverse Graphics GANs}
\end{center}


\section*{Problem description}
Inverse graphics GANs are used to decode 3D data from 2D data. This means that they can be used for converting 2D images into 3D digital objects that can be used for as 3D assets. The problem is still more or less unsolved as we are trying to reverse an irreversible operation, i.e. moving from a 2 dimensional data space to a 3 dimensional one. Therefore, we can make use or randomness that EA's inherently have. This can allow us to attempt to optimize the GAN to produce more accurate results. Therefore, the problem statement simply is to use EA to optimize an inverse graphics GAN to produce higher quality 3D objects from 2D data.

\section*{CI techniques}
We would be using Genetic Algorithms for the optimization.

\section*{Final outcome \& visualization}
The final deliverable would include a report detailing our findings from running the algorithm and the outcome acquired as a result of it. There might also be a visualzation of the process of the conversion from a 2D image to 3D image.

\section*{Datasets} 
The dataset in use will be;
\begin{itemize}
    \item https://paperswithcode.com/dataset/imagenet.
    \item Another dataset would be a generic dataset of images of Chanterelle mushrooms.
    \item Some modles will be used from ShapeNet dataset to evaluate our modles (https://shapenet.org/model-querier).
\end{itemize}

\begin{reference}
  \bibent
  Lunz, Sebastian., Li, Yinghzen., Fitzgibbon, Andrew., and Kushman, Nate. \textit{Inverse Graphics GAN: Learning to Generate 3D Shapes from Unstructured 2D Data} CoRR, 2020. Print.

  \bibent
  Chaoyue Wang, Chang Xu, Xin Yao, and Dacheng Tao. \textit{Evolutionary Generative Adversarial Networks} CoRR, 2018. Print.
\end{reference}
\end{document}
